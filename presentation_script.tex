\documentclass[12pt,a4paper]{article}
\usepackage[utf8]{inputenc}
\usepackage[T1]{fontenc}
\usepackage[polish]{babel}
\usepackage{geometry}
\usepackage{enumitem}
\usepackage{xcolor}
\usepackage{titlesec}

\geometry{margin=2.5cm}
\setlength{\parindent}{0pt}
\setlength{\parskip}{10pt}

\titleformat{\section}[block]{\Large\bfseries\color{blue!70!black}}{Slajd \thesection:}{0.5em}{}
\titleformat{\subsection}[block]{\large\bfseries\color{gray}}{}{0em}{}

\title{\huge\textbf{Skrypt do prezentacji}\\[0.5cm]
\Large Predykcja Opóźnień Lotów - Uczenie Maszynowe}
\author{K. Arkit, D. Chomiak, Ł. Guziczak, D. Sobótka}
\date{\today}

\begin{document}

\maketitle
\vspace{1cm}

\textit{Ten dokument zawiera pełny skrypt do wygłoszenia prezentacji. Każda sekcja odpowiada jednemu slajdowi. Czas prezentacji: 10-15 minut.}

\newpage

\section{Slajd tytułowy}
\subsection{Czas: 30 sekund}

Dzień dobry Państwu. Nazywamy się Kamil Arkit, Dawid Chomiak, Łukasz Guziczak i Dawid Sobótka. 

Dzisiaj przedstawimy Państwu nasz projekt z uczenia maszynowego dotyczący predykcji opóźnień lotów. Jest to problem, który dotyka miliony pasażerów rocznie i generuje ogromne koszty dla linii lotniczych.

Przejdźmy do definicji problemu.

\section{Definicja problemu}
\subsection{Czas: 2 minuty}

Naszym celem było stworzenie modelu uczenia maszynowego, który przewidzi, czy lot będzie opóźniony o więcej niż 15 minut. Jest to klasyczny problem klasyfikacji binarnej.

Wykorzystaliśmy dataset US Flight Delays z 2015 roku, który zawiera informacje o 5,8 miliona lotów. Ze względu na ograniczenia obliczeniowe, pracowaliśmy na próbce 500 tysięcy rekordów, zachowując reprezentatywność danych.

Dataset zawiera 31 atrybutów opisujących każdy lot - od podstawowych informacji jak linia lotnicza i lotniska, przez szczegóły czasowe, aż po dane o dystansie.

Po prawej stronie widzą Państwo rozkład opóźnień w naszych danych. Kluczowa obserwacja: tylko 19,4% lotów było opóźnionych o więcej niż 15 minut. To oznacza, że mamy do czynienia z niezbalansowanym datasetem, co stanowi jedno z głównych wyzwań w tym projekcie.

Dlaczego wybraliśmy ten problem? Po pierwsze, jest to duży, rzeczywisty dataset, który pozwala na praktyczne zastosowanie technik uczenia maszynowego. Po drugie, niezbalansowane klasy wymagają przemyślanego podejścia do modelowania. Po trzecie, możemy przetestować różnorodne techniki - od prostych modeli po zaawansowane ensemble.

\section{Ewolucja modelu - 4 etapy}
\subsection{Czas: 2,5 minuty}

Na tym slajdzie przedstawiamy ewolucję naszego modelu przez cztery kluczowe etapy. Wykres pokazuje, jak zmieniały się główne metryki w kolejnych iteracjach.

\textbf{Etap 1 - Baseline:} Zaczęliśmy od podstawowego modelu z tylko 12 cechami - były to surowe atrybuty z datasetu jak miesiąc, dzień tygodnia, linia lotnicza czy dystans. Bez żadnego feature engineeringu. Jak widać, model osiągnął niski recall - zaledwie 14,5%. Wykrywał tylko co siódme opóźnienie. To był nasz punkt startowy.

\textbf{Etap 2 - Data Leakage:} W drugim etapie celowo wprowadziliśmy błąd data leakage, dodając cechę DELAY\_LOG, która zawierała informacje dostępne dopiero po fakcie. Metryki wystrzeliły w górę - recall 100%, F1-score 1,0. To był ważny eksperyment edukacyjny, pokazujący jak łatwo można otrzymać "zbyt dobre" wyniki i jak ważna jest walidacja modelu.

\textbf{Etap 3 - Fast Optimized:} Po usunięciu wycieku danych, skupiliśmy się na prawdziwej optymalizacji. Zastosowaliśmy ensemble trzech modeli - Random Forest, XGBoost i LightGBM. Dodatkowo usunęliśmy ekstremalne wartości opóźnień powyżej 300 minut. Model osiągnął recall 55,1% i F1-score 0,483.

\textbf{Etap 4 - Final Model:} W ostatnim etapie naprawiliśmy błąd z usuwaniem outlierów - zachowaliśmy WSZYSTKIE dane, również ekstremalne opóźnienia. Rozszerzyliśmy liczbę cech do 28 poprzez feature engineering. Co to znaczy? Stworzyliśmy nowe, bardziej użyteczne zmienne: cykliczne kodowanie czasu (godzina jako sinus i cosinus, żeby model wiedział że 23:00 jest blisko 1:00), binarne flagi jak IS\_RUSH\_HOUR czy IS\_WEEKEND, wskaźniki opóźnień dla linii i lotnisk, a nawet interakcje między cechami - na przykład łącząc godziny szczytu z ryzykiem konkretnej linii lotniczej. Finalny model osiągnął F1-score 0,491 i ROC-AUC 0,769.

Warto zauważyć spadek recall z 55,1% do 54,4% po zachowaniu outlierów - to pokazuje, jak trudne do przewidzenia są ekstremalne opóźnienia.

\section{Porównanie modeli - krzywe ROC}
\subsection{Czas: 2 minuty}

Ten wykres przedstawia krzywe ROC dla wszystkich czterech etapów naszego projektu. Krzywa ROC pokazuje zależność między true positive rate a false positive rate dla różnych progów decyzyjnych.

Spójrzmy na ewolucję:
- Niebieski Baseline z AUC 0,643 - ledwo lepszy niż losowy klasyfikator
- Pomarańczowa krzywa Data Leakage - praktycznie idealna z AUC 0,999, co było ostrzeżeniem o problemie
- Zielona Fast Optimized z AUC 0,751 - znacząca poprawa po usunięciu wycieku
- Czerwony Final Model z AUC 0,769 - nasz najlepszy uczciwy wynik

Kluczowa obserwacja: mimo że w Final Model mieliśmy niższy recall niż w Fast Optimized, to ROC-AUC jest wyższy. To pokazuje, że model lepiej radzi sobie z rankingiem prawdopodobieństw, co daje większą elastyczność w doborze progu decyzyjnego w zależności od potrzeb biznesowych.

Etap z Data Leakage był cenną lekcją - pokazał nam, jak ważne jest dokładne sprawdzenie, czy nasze cechy są dostępne w momencie predykcji. W prawdziwym świecie taki błąd mógłby kosztować miliony.

\section{Analiza najważniejszych cech}
\subsection{Czas: 2 minuty}

Przejdźmy do analizy, które cechy okazały się najważniejsze dla naszego modelu.

Wykres po lewej pokazuje top 15 cech według ważności w modelu XGBoost. Dominują cechy związane z czasem:

\textbf{IS\_RUSH\_HOUR} - najważniejsza cecha z wagą 16,5%. To binarna zmienna określająca, czy lot odbywa się w godzinach szczytu (6-9 rano i 16-19 po południu). Intuicyjnie ma to sens - w godzinach szczytu lotniska są bardziej zatłoczone.

\textbf{HOUR\_SIN i HOUR\_COS} - to cykliczne kodowanie godziny odlotu. Używamy funkcji sinus i cosinus, aby model zrozumiał, że godzina 23:00 jest blisko 1:00. Razem te cechy mają ponad 20% ważności.

\textbf{IS\_WEEKEND i IS\_FRIDAY} - weekendy mają inny wzorzec ruchu niż dni robocze. Ciekawe, że piątek został wyodrębniony jako osobna cecha.

Wśród innych ważnych cech mamy:
- Linię lotniczą (8,8%) - niektóre linie są bardziej punktualne
- Lotnisko wylotu (7,9%) - duże huby mają większe ryzyko opóźnień
- Dystans lotu (7,6%) - dłuższe loty mają więcej czasu na nadrobienie opóźnień

Główny wniosek: czas odlotu ma największy wpływ na przewidywanie opóźnień. To sugeruje, że harmonogram lotów i obciążenie lotnisk w różnych porach dnia są kluczowe dla punktualności.

\section{Wydajność finalna}
\subsection{Czas: 2 minuty}

Na tym slajdzie przedstawiamy szczegółową analizę wydajności naszego finalnego modelu.

Po lewej widzimy macierz pomyłek. Z prawie 100 tysięcy lotów w zbiorze testowym:
- Poprawnie przewidzieliśmy 66,600 lotów na czas (True Negatives)
- Wykryliśmy 12,400 opóźnionych lotów (True Positives)
- Przegapiliśmy 10,400 opóźnień (False Negatives) - to nasz główny problem
- Fałszywie alarmowaliśmy w 15,100 przypadkach (False Positives)

Wykres po prawej pokazuje kluczowe odkrycie: recall dramatycznie spada wraz z wielkością opóźnienia. Dla małych opóźnień (15-30 minut) wykrywamy 61%, ale dla ekstremalnych opóźnień powyżej 300 minut tylko 39,6%. To sugeruje, że ekstremalne opóźnienia są spowodowane nieprzewidywalnymi zdarzeniami jak awarie czy ekstremalna pogoda.

Nasze finalne metryki to:
- F1-score: 0,491
- Recall: 54,4% - wykrywamy nieco ponad połowę opóźnień
- Precision: 45,0% - prawie połowa naszych alarmów to fałszywe alarmy
- ROC-AUC: 0,769 - dobra zdolność do rankingu

Te wyniki pokazują klasyczny trade-off w uczeniu maszynowym między recall a precision. W kontekście biznesowym należałoby zdecydować, co jest gorsze: przegapione opóźnienie czy fałszywy alarm.

\section{Podsumowanie}
\subsection{Czas: 2 minuty}

Podsumowując nasz projekt:

\textbf{Co udało się osiągnąć:}
Przetestowaliśmy cztery różne podejścia, od prostego baseline po zaawansowany ensemble. Zidentyfikowaliśmy i naprawiliśmy problem data leakage, co było cenną lekcją. Poprzez feature engineering zwiększyliśmy liczbę cech z 12 do 28, włączając w to cykliczne kodowanie czasu i interakcje między zmiennymi. Najlepszy model - XGBoost - osiągnął ROC-AUC 0,769, co jest solidnym wynikiem dla tego problemu.

\textbf{Napotkane wyzwania:}
Głównym wyzwaniem były niezbalansowane klasy - tylko 19,4% lotów było opóźnionych. Musieliśmy balansować między recall a precision - więcej wykrytych opóźnień oznaczało więcej fałszywych alarmów. Szczególnie problematyczne okazały się ekstremalne opóźnienia powyżej 300 minut. Dodatkowo, brak danych o warunkach pogodowych ograniczał możliwości modelu.

\textbf{Propozycje ulepszeń:}
Widzimy kilka kierunków rozwoju:
- Model dwuetapowy: najpierw klasyfikacja normalne/ekstremalne opóźnienie, potem dedykowane modele
- Dane pogodowe znacząco poprawiłyby predykcje
- Stacking ensemble z meta-learnerem mógłby lepiej łączyć różne modele
- Kalibracja prawdopodobieństw dałaby bardziej wiarygodne estymaty ryzyka

Ten projekt pokazał nam złożoność rzeczywistych problemów ML i znaczenie iteracyjnego podejścia do modelowania.

\section{Dziękujemy za uwagę!}
\subsection{Czas: 30 sekund}

Dziękujemy Państwu za uwagę i zapraszamy do zadawania pytań.

Cały kod źródłowy naszego projektu jest dostępny na GitHubie w repozytorium ML-FlightDelaysAndCancellations.

Jesteśmy gotowi odpowiedzieć na wszelkie pytania dotyczące metodologii, wyników czy potencjalnych zastosowań naszego modelu.

\vspace{2cm}
\begin{center}
\rule{0.8\textwidth}{0.5pt}
\end{center}

\textbf{Dodatkowe informacje na wypadek pytań:}

\begin{itemize}
\item \textbf{Dlaczego 15 minut?} - To standard branżowy dla definicji opóźnienia w USA
\item \textbf{Dlaczego XGBoost?} - Najlepsza równowaga między wydajnością a czasem treningu
\item \textbf{Koszt biznesowy?} - Średni koszt opóźnienia to \$75/minuta dla linii lotniczej
\item \textbf{Czas treningu?} - Final model: około 15 sekund na 500k rekordów
\item \textbf{Możliwość wdrożenia?} - Tak, model jest gotowy do deploymentu, ROC-AUC 0,769 jest akceptowalny
\end{itemize}

\end{document}